\documentclass{beamer}
% \usetheme{Boadilla}
\usetheme{Copenhagen}

\usepackage[utf8]{inputenc}

\usepackage[bibstyle=ieee,citestyle=numeric-comp]{biblatex}
\usepackage[title, titletoc]{appendix} %improved appendix controls

\usepackage{amsmath} %MATHS
\usepackage{amssymb}

\usepackage{float} %float placement
\usepackage[font={small}, justification=centering]{caption}
\usepackage{subcaption} % subfigure package
\usepackage[export]{adjustbox}
\usepackage{pdfpages} %integrate other pdf's 
\usepackage{graphicx} %image insertion
\usepackage[newfloat]{minted} %auto code formatting for multiple languages
% \usepackage{enumitem}
\usepackage{multirow} % for table formatting

\usepackage{color,soul} % used for highlighting
\usepackage{xcolor}	%more colors
% \usepackage{sectsty}

% \usepackage{titlesec} # breaks for presentation
% \usepackage{todonotes} %reminders
\usepackage{blindtext} % text for document formatting
\usepackage{comment} %block comments
% \usepackage[firstpage=true]{background}

% \usepackage[hidelinks, breaklinks]{hyperref} %improved referencing; already included for presentations
\usepackage[capitalise, nameinlink, noabbrev, english]{cleveref}

%extra set-up
% !TeX spellcheck = en_GB
\graphicspath{{images/}}
\addbibresource{bibliography.bib}

% environments and functions
\newenvironment{code}{\captionsetup{type=listing}}{} % code environment that can page-break
\SetupFloatingEnvironment{listing}{name=Source Code}
\definecolor{bg}{rgb}{0.95,0.95,0.95}


\title{LaTeX Tutorial}
\author{Jacob White}
\institute{Memorial University of Newfoundland}
\date{May 7, 2021}

\begin{document}

\begin{frame}
    \titlepage
\end{frame}

% \begin{frame}{Table of Contents}
%     \tableofcontents
% \end{frame}

\section{Starting Notes}

\begin{frame}{Scope}

    \begin{itemize}
        \item This will be more of a starter to give you the gist of how to get going
        \item Lots of nuance - not feasible to teach all of the ins and outs of \LaTeX
    \end{itemize}
    
\end{frame}

\subsection{Useful Resources}
\begin{frame}{Editors}

    \textbf{RECOMMENDATION}\\
    \begin{itemize}
        \item Overleaf (online): \url{https://www.overleaf.com}
    \end{itemize}
    \vspace{\baselineskip}
    
    \textbf{Others} (\textit{substantially} more inconvenient to set up!)\\
    \begin{itemize}
        \item MiKTeX: \url{https://miktex.org/}
        \item Visual Studio Code w/ LaTeX Workshop extension
    \end{itemize}
    
\end{frame}

\begin{frame}{Revision and Comments}

    \textbf{GitHub}
    \begin{itemize}
        \item A repository can be used for tracking changes between people (us and Sarah) 
        \item Built in functionality with Overleaf
    \end{itemize}
    
    \vspace{\baselineskip}
    
    \textbf{Grammarly}
    \begin{itemize}
        \item Can help you proof/revise your writing quickly
        \item \url{https://www.engr.mun.ca/techcomm/wp-content/uploads/2020/09/GrammarlyPremiumInstructions.pdf}
    \end{itemize}
    
\end{frame}

\begin{frame}{Examples}

    \begin{itemize}
        \item Easiest way to get started making \LaTeX{} documents is to bootstrap yours from examples
        \item I'll send out an email with some of my previous work that you can pick through
    \end{itemize}
    
\end{frame}

\subsection{Golden Rule}

\begin{frame}{When your unsure how to do something in \LaTeX... }
    
    \begin{itemize}
        \item Google your question, followed by `latex'
        \item This works 95\% of the time, Latex is old so there are lots of answers out there!
        \item The Tex stack exchange is a great source too: \url{https://tex.stackexchange.com/}
    \end{itemize}
    
\end{frame}


\section{The Basics}
\subsection{Syntax}

\begin{frame}{The \LaTeX `Language'}
    
    \begin{itemize}
        \item You'll notice many similarities to coding languages
        \item Using latex, you structure how the document looks using \textit{commands} and \textit{environments}
        \item Additional features can be added using \textit{packages}
        \item We'll learn how to use all of these by making an document throughout this presentation
    \end{itemize}
    
\end{frame}

\begin{frame}[fragile]{Commands}

    The syntax for using commands is as follows:

    \begin{minted}{tex}
\commandname[optional arguments]{required arguments}
    \end{minted}
    \vspace{\baselineskip}
    
    Some examples:
    \begin{minted}{tex}
\tableofcontents
\pagenumbering{arabic}
\usepackage[bibstyle=ieee]{biblatex}
    \end{minted}
    
\end{frame}

\begin{frame}[fragile]{Environments}

    Used to encapsulate specific formatting and regions of the document, such as figures, tables, math, etc.
    \vspace{\baselineskip}
    
    \begin{minted}{tex}
\begin{document}
    ...
\end{document}
    \end{minted}
    \vspace{\baselineskip}
        
    You can also used a stared (*) version if you don't want it to be numbered
    \vspace{\baselineskip}
    
    \begin{minted}{tex}
\begin{equation*}
    ...
\end{equation*}
    \end{minted}
    
\end{frame}

\subsection{Preamble Essentials}
\begin{frame}[fragile]{Document Classes}
    
    \begin{itemize}
        \item The first thing in any document
        \item Many types:
        \begin{itemize}
            \item article (most common)
            \item pdfathesis (for writing thesis)
            \item beamer (for presentations, i.e. this!)
            \item etc.
        \end{itemize}
    \end{itemize}
    \vspace{\baselineskip}
    
    \begin{minted}{tex}
  \documentclass{beamer}
  \documentclass[12pt]{article}
    \end{minted}
    
\end{frame}

\begin{frame}[fragile]{Packages}
    As mentioned, packages are how you can add additional functionality to \LaTeX\\
    
    For example, here is how you can alter the proportions of you paper:
    \vspace{\baselineskip}
    
    \begin{minted}{tex}
\usepackage[paper=letterpaper, margin=1in]{geometry}
    \end{minted}
    
\end{frame}


\section{Writing your document}
\begin{frame}[fragile]{Beginning your Doc}

    So far, everything shown has been preamble - things will only show up once you put in:
    \vspace{\baselineskip}

    \begin{minted}{tex}
\begin{document}
    ...
\end{document}
    \end{minted}
    \vspace{\baselineskip}
    
    The entire written document must be within this environment
    
\end{frame}

\subsection{Basic Writing}
\begin{frame}[fragile]{Sectioning}
    To divide up your document, we use \textit{sections}. You can further subdivide using \textit{subsections}, and \textit{subsubsections}.
    \vspace{\baselineskip}
    
    \begin{minted}{tex}
- \section{Section Name}
- \subsection{This is a subsection}
- \subsubsection{This is a subsubsection}
    \end{minted}
    \vspace{\baselineskip}
    
    \begin{figure}[H]
        \centering
        \includegraphics{images/sections.png}
    \end{figure}
    
\end{frame}

\begin{frame}[fragile]{Writing}
    
    There's nothing too complicated when it comes to writing actual words down in latex, you just have to have them placed in the correct \textbackslash{}section\{\} of the document.\\
    
    There are some restricted characters that can't be typed on their own such as \_ \% \& \textbackslash{} \{ \} \$, and need to be typed like so,\\
    
    \begin{minted}{tex}
\_ \% \& \textbackslash \{ \} \$
    \end{minted}
    
\end{frame}

\begin{frame}{Writing}

    To end a line use a double-backslash, \textbackslash\textbackslash \\ 
    
    To end a paragraph, use \textbackslash{}par \\
    
    To add a space between lines or paragraphs, you have to and a black new-line after the \textbackslash\textbackslash
    
\end{frame}

\begin{frame}[fragile]{Referencing Objects}

    Oftentimes, you'll want to refer to figures or tables within your document. This is easily done in latex as it automatically orders and numbers these objects for you.\\
    
    There are a few packages that improve the default object referencing that I recommend:
    \vspace{\baselineskip}
    
    \begin{minted}[fontsize=\footnotesize, breaklines]{tex}
\usepackage[hidelinks, breaklinks]{hyperref}
\usepackage[capitalise, nameinlink, noabbrev, english]{cleveref}
    \end{minted}
    
\end{frame}

\begin{frame}[fragile]{Referencing Objects}

        To make a referencing point, use the \textbackslash{}label\{labelName\} command (many examples coming up).\\
        
        To refer back to this point, use \textbackslash{}cref\{labelName\}.
        \begin{itemize}
            \item Say you are referencing a figure, the result will be: `Figure 7'
        \end{itemize}
        \vspace{\baselineskip}
        
        If you don't want the `Figure' part to appear, use the \textbackslash{}ref\{\} command instead
    
\end{frame}

\subsection{Figures}
\begin{frame}[fragile]{Figures Packages}

    The following packages and commands are required:
    \vspace{\baselineskip}

    \begin{minted}[breaklines, fontsize=\footnotesize]{tex}
\usepackage{graphicx} %for images
\graphicspath{{images/}} % folder for images
\usepackage[font={small}, justification=centering]{caption} % caption style
\usepackage{float} %float placement
\usepackage{subcaption} % for subfigures 
    \end{minted}
    
\end{frame}

\begin{frame}{Figures}

    As seen in \cref{fig:eg_figure_label}, you can add in images using figures.

    \begin{figure}[H]
        \centering
        \includegraphics[scale=0.1]{images/example_image.JPG}
        \caption{The caption goes in here.}
        \label{fig:eg_figure_label}
    \end{figure}
    
\end{frame}

\begin{frame}[fragile]{Figure structure}
    
    To make the figure in the previous slide, the below was used:
    \vspace{\baselineskip}
    
    \begin{minted}[fontsize=\small]{tex}
\begin{figure}[htbH]
    \centering
    \includegraphics[scale=0.1]{images/example_image.JPG}
    \caption{The caption goes in here.}
    \label{fig:eg_figure_label}
\end{figure}
    \end{minted}
    
\end{frame}

\begin{frame}{Sub-Figures}
    
    You can also have multiple images per figure if they are related, such as \cref{fig:eg_sub_figure}.

    \begin{figure}[H]
        \centering
        \begin{subfigure}[c]{0.29\textwidth}
            \includegraphics[width=\textwidth]{images/example_image.JPG}
            \caption{}
        \end{subfigure}
        \begin{subfigure}[c]{0.3\textwidth}
            \includegraphics[width=\textwidth]{images/example_image_2.jpg}
            \caption{}
        \end{subfigure}
        \caption{(a) A cat in a hat. (b) A cat in a box.}
        \label{fig:eg_sub_figure}
    \end{figure}
    
\end{frame}

\begin{frame}[fragile]{Sub-Figures structure}
    
    \begin{minted}[breaklines, fontsize=\small]{tex}
\begin{figure}[H]
    \centering
    \begin{subfigure}[c]{0.29\textwidth}
        \includegraphics[width=\textwidth]{ images/example_image.JPG}
        \caption{}
    \end{subfigure}
    \begin{subfigure}[c]{0.3\textwidth}
        \includegraphics[width=\textwidth]{ images/example_image_2.jpg}
        \caption{}
    \end{subfigure}
    \caption{(a) A cat in a hat. (b) A cat in a box.}
    \label{fig:eg_sub_figure}
\end{figure}
    \end{minted}
    
\end{frame}

\subsection{Tables}
\begin{frame}{Tables}
    
    Latex can format your tables, like the example seen in \cref{tab:table_eg}
    
    \begin{table}[H]
        \centering
        \begin{tabular}{c|c|c}
            Order & PFE Error [dB] & LS Error [dB] \\ \hline
            (50/50) & 1.97972e-10 & 1.95831e-10 \\ \hline
            (100/100) & 0.00761004 & 7.18605e-09 \\ \hline
            (150/150) & 4.17774 & 8.00024e-05 \\ \hline
            (200/200) & 4.85818 & 9.27059e-09 \\ \hline
            (1000/1000) & 4.44133 & 1.00683e-07 \\ \hline
        \end{tabular}
        \caption{The mean absolute dB error for the \textbf{room response} for various filter orders.}
        \label{tab:table_eg}
    \end{table}
    
\end{frame}

\begin{frame}[fragile]{Table Structure}

    \begin{minted}[breaklines, fontsize=\small]{tex}
\begin{table}[H]
    \centering
    \begin{tabular}{c|c|c}
        Order & PFE Error [dB] & LS Error [dB] \\ \hline
        (50/50) & 1.97972e-10 & 1.95831e-10 \\ \hline
        (100/100) & 0.00761004 & 7.18605e-09 \\ \hline
        (150/150) & 4.17774 & 8.00024e-05 \\ \hline
        (200/200) & 4.85818 & 9.27059e-09 \\ \hline
        (1000/1000) & 4.44133 & 1.00683e-07 \\ \hline
    \end{tabular}
    \caption{The mean absolute dB error for the \textbf{room response} for various filter orders.}
    \label{tab:table_eg}
\end{table}
    \end{minted}
    
\end{frame}

\subsection{Math}
\begin{frame}[fragile]{Math Packages}
    Two incredibly useful packages for math formatting in LaTeX:
    \vspace{\baselineskip}
    
    \begin{minted}{tex}
\usepackage{amsmath}
\usepackage{amssymb}
    \end{minted}
    
\end{frame}

\begin{frame}[fragile]{Ways to write math}
    Math can be written inline and in block environments: \\
    
    For \textbf{inline}, surround the math in \$: \\
    
    \begin{minted}[fontsize=\small]{tex}
Radial frequency is, $\Omega = 2\pi f$
For proper sampling, $f_{s} > 2\cdot f_{m}$
    \end{minted}
    \vspace{\baselineskip}
    
    Radial frequency is, $\Omega = 2\pi f$ \\
    For proper sampling, $f_{s} > 2\cdot f_{m}$ \\
    
\end{frame}

\begin{frame}[fragile]{The Math Environment}
    
    For standalone equations use the \textit{equation} environment, like in \cref{eqn:eg_equation}:
    
    \begin{equation}
        H(z^{-1}) = \sum_{l=1}^{L} \frac{ b_{0,l} + b_{1,l}z^{-1} }{ 1 + a_{1,l}z^{-1} + a_{2,l}z^{-2} } + \sum_{k=0}^{K}f_{k}z^{-k}
        \label{eqn:eg_equation}
    \end{equation}
    
    \begin{minted}[breaklines, fontsize=\small]{tex}
\begin{equation}
    H(z^{-1}) = \sum_{l=1}^{L} \frac{b_{0,l} + b_{1,l}z^{-1}}{1 + a_{1,l}z^{-1} + a_{2,l}z^{-2}} + \sum_{k=0}^{K}f_{k}z^{-k}
    \label{eqn:no_del_par}
\end{equation}
    \end{minted}
    
\end{frame}

\begin{frame}[fragile]{Multi-Line Equations}

    Equations can be broken into multiple lines, like in \cref{eqn:eg_split}:
    
    \begin{equation}
        \begin{split}
            y + 2x & = 7x + 5 \\
            y & = 5x + 5 \\
            y & = 5(x+1)
        \end{split}
        \label{eqn:eg_split}
    \end{equation}
    
    \begin{minted}[fontsize=\small, baselinestretch=1]{tex}
\begin{equation}
    \begin{split}
        y + 2x & = 7x + 5 \\
        y & = 5x + 5 \\
        y & = 5(x+1)
    \end{split}
    \label{eqn:eg_split}
\end{equation}
    \end{minted}
    
\end{frame}

\begin{frame}{Notes about math usages}
    
    \begin{itemize}
        \item The math environments are what you can use for subscripts and superscripts in text, eg. May $7^{th}, 2021$
        \item You can only use the Greek characters within math environments, eg. $\alpha, \beta, ..., \omega$
    \end{itemize}
    
\end{frame}

\subsection{Other Environments}
\begin{frame}{Lists}

    \begin{itemize}
        \item For unnumbered lists (like this), use the \textit{listing} environment
        \item (Unnumbered list item)
    \end{itemize}
    
    \vspace{\baselineskip}

    \begin{enumerate}
        \item For numbered lists (like \textit{this}), use the \textit{enumerate} environment
        \item (Numbered list item)
    \end{enumerate}

\end{frame}

\section{Citations and Bibliography Management}
\subsection{Bibtex Files}
\begin{frame}[fragile]{Using Bibtex for Bibliography Management}
    Possibly the best reason to start using Latex! Bibtex streamlines citations and bibliography's in your work.\\
    
    To make use of it, we need the following package:
    \vspace{\baselineskip}
    
    \begin{minted}{tex}
\usepackage[bibstyle=ieee]{biblatex}
    \end{minted}
    \vspace{\baselineskip}
    
    There are lots of different citation styles that can be used (IEEE, MLA, APA, ...)
    
\end{frame}

\begin{frame}{Bib Files - Where to get them?}

    \begin{figure}
        \centering
        \includegraphics[scale=0.8]{images/google_scholar.png}
        \caption{You can get the .bib files directly from google scholar}
        \label{fig:bib_google}
    \end{figure}
    
    \begin{figure}
        \centering
        \includegraphics[scale=0.5]{images/mun_lib.png}
        \caption{You can also get them from the MUN online libraries}
        \label{fig:bib_mun}
    \end{figure}
    
\end{frame}

\begin{frame}[fragile]{Bib File Format}
    
    \begin{minted}[fontsize=\scriptsize, breaklines]{tex}
@article{lotte_review_2018,
	title = {A review of classification algorithms for {EEG}-based brain–computer interfaces: a 10 year update},
	volume = {15},
	issn = {1741-2560, 1741-2552},
	shorttitle = {A review of classification algorithms for {EEG}-based brain–computer interfaces},
	...
}
    \end{minted}
    
\end{frame}

\begin{frame}[fragile]{Bib Files}

    When you gather all of the bibtex files that you need, add them all together to a file that you specify:
    \vspace{\baselineskip}
    
    \begin{minted}{tex}
\addbibresource{bibliography.bib}
    \end{minted}
    
\end{frame}

\subsection{Citations and Bibliographies}
\begin{frame}[fragile]{Including Citations}

    Once your bibtex references are gathered and set up, citing others is easy:
    \vspace{\baselineskip}

    \begin{minted}{tex}
\cite{reference_tag}
    \end{minted}
    \vspace{\baselineskip}
    
    Despite advances in parallel computing technology, the most common approach to convert filters to a parallel form has some inherent issues \cite{BankBalazs2018CIIR}. 
        
\end{frame}

\begin{frame}[fragile]{Adding your Bibliography}
    
    Adding in the bibliography is just as easy:
    \vspace{\baselineskip}
    
    \begin{minted}{tex}
\printbibliography
    \end{minted}
    \vspace{\baselineskip}
    
    Everything that you reference using the \textbackslash{}cite\{\} command will be added to the bibliography, formatted as per the settings at the package inclusion.
    
\end{frame}

\begin{frame}{The Bibliography}

    \begin{figure}
        \centering
        \includegraphics[width=\linewidth]{images/bibliography.png}
    \end{figure}
    
\end{frame}

\section{Final Notes}
\begin{frame}{Final Notes}
    With all of the examples that I'll send out, there will also be a long-form latex tutorial pdf\\
    
    Don't forget the golden rule: the internet has (nearly) all of the answers out there already!\\
    
    Overleaf has lots of pre-made templates that can be useful, here's one for a MUN thesis:
    \url{https://www.overleaf.com/gallery/tagged/mun}
    
\end{frame}

\begin{frame}{}
    Thanks for your time!
\end{frame}

\end{document}
